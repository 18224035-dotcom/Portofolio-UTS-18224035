\hypersetup{
  pdftitle={Identitasku},
  pdfauthor={18224035 Stella Cometta Febriana},
  colorlinks=true,
  linkcolor={blue},
  filecolor={Maroon},
  citecolor={Blue},
  urlcolor={Blue},
  pdfcreator={LaTeX via pandoc}}

\title{Identitasku}
\subtitle{Portfolio Asesmen II-2100 Komunikasi Interpersonal dan Profesional}
\author{18224035 Stella Cometta Febriana}
\date{2025-10-23}

\begin{document}
\maketitle

\renewcommand*\contentsname{Daftar Isi}
{
\hypersetup{linkcolor=}
\setcounter{tocdepth}{2}
\tableofcontents
}

\bookmarksetup{startatroot}

\chapter*{Identitasku}\label{identitasku}
\addcontentsline{toc}{chapter}{Identitasku}

\markboth{Identitasku}{Identitasku}

\begin{figure}[H]
{\centering \includegraphics[width=8cm]{images/stella.jpg}}
\caption{Stella Cometta Febriana}
\end{figure}

Halo, aku **Stella Cometta Febriana**, mahasiswa **Sistem dan Teknologi Informasi ITB**.  
Aku bukan orang yang pendiam — aku suka belajar hal baru dan selalu penasaran dengan cara sesuatu bekerja.  
Bagiku, setiap hal baru yang kupelajari adalah langkah kecil untuk memahami dunia dan diriku sendiri.

Aku tertarik pada bagaimana teknologi bisa membantu manusia hidup lebih baik, bukan hanya lebih cepat.  
Setiap sistem atau proyek yang kugarap selalu kupertanyakan: *“Apakah ini benar-benar bermanfaat untuk orang lain?”*  
Dari situlah aku belajar bahwa logika dan empati bisa berjalan beriringan.

Di luar kampus, aku suka membaca, menulis, dan mencoba hal-hal baru — entah itu eksperimen kecil di dapur, atau proyek iseng dengan baris-baris kode.  
Aku juga belajar untuk tidak terlalu keras pada diri sendiri. Kadang aku masih suka membandingkan diri dengan orang lain,  
tapi sekarang aku tahu, setiap orang punya waktunya masing-masing, dan tidak apa-apa jika jalanku sedikit lebih lambat, asalkan aku terus berjalan.

Inilah alasanku menulis portofolio ini — bukan untuk menunjukkan kesempurnaan,  
tapi untuk mencatat prosesku mengenal diri sendiri dan terus belajar jadi lebih baik.

---

\bookmarksetup{startatroot}


\chapter{UTS-1 Aku, di Antara Kode dan Kopi Pagi}\label{uts-1}

% (isi lengkap UTS-1 tetap sama seperti sebelumnya)

---

\chapter{UTS-2 Lagu Kecil untuk Kamu yang Sedang Berjuang}\label{uts-2}
% (isi lengkap UTS-2 seperti sebelumnya)

---

\chapter{UTS-3 Tentang Gagal, dan Kenapa Aku Tidak Berhenti}\label{uts-3}
% (isi lengkap UTS-3 seperti sebelumnya)

---

\chapter{UTS-4 Bentukku: Antara Logika dan Empati}\label{uts-4}
% (isi lengkap UTS-4 seperti sebelumnya)

---

\chapter{UTS-5 Menemukan Diri Lewat Komunikasi}\label{uts-5}
% (isi lengkap UTS-5 seperti sebelumnya)

\bookmarksetup{startatroot}

\chapter*{Referensi}\label{referensi}
\addcontentsline{toc}{chapter}{Referensi}

\markboth{Referensi}{Referensi}
\phantomsection\label{refs}

\end{document}
